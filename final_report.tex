\documentclass[10pt]{IEEEtran}
\IEEEoverridecommandlockouts
% The preceding line is only needed to identify funding in the first footnote. If that is unneeded, please comment it out.
\usepackage{cite}
\usepackage{amsmath,amssymb,amsfonts}
\usepackage{algorithmic}
\usepackage{graphicx}
\usepackage{textcomp}
\def\BibTeX{{\rm B\kern-.05em{\sc i\kern-.025em b}\kern-.08em
    T\kern-.1667em\lower.7ex\hbox{E}\kern-.125emX}}
\begin{document}

\title{Visualization Dashboard for Rigidity Analysis of Protein Mutations}

\author{\IEEEauthorblockN{Dylan Carpenter, Josh Dombal, Hunter Read}\\
\IEEEauthorblockA{\textit{Western Washington University} \\
Bellingham, WA, USA}
}

\maketitle

\begin{abstract}
Analyzing the effects of protein mutation is crucial to future treatments and development of novel medications for some diseases. However wet lab experiments can be time consuming and exhaustive screenings of protein mutations impossible. To this end, several tools, and computational software can guide this research, but may not provide overall insights into all possible protein mutations. For this, we have developed a visualization dashboard that enables easier insights into the effects of protein mutation through rigidity analysis, and is an effective tool in exploration and identification of possible critical mutations. We conclude that our visualization dashboard allows for novel insight into exhaustive protein mutation data.\\
\end{abstract}

\begin{IEEEkeywords}
bioinformatics, mutation, protein, rigidity, visualization
\end{IEEEkeywords}

\section{Introduction}
Gaining insight into protein mutations is important for researchers and pharmaceutical companies looking to develop new treatments and medications. With some insight into mutant proteins, we could advance cancer diagnosis, prognosis, and therapies \cite{b2}, develop novel medications for Fabry disease, or further a general understanding into many illnesses that plague civilization. Current techniques allow experimentalist to mutate and analyze these variations in a wet lab, but can require months of work with only the hope of providing the information scientists may need\cite{b3}. But due to these time requirements, thorough and exhaustive protein mutation screenings are difficult for small proteins and impossible for large proteins, arising a need for in-silico protein mutation and analysis.\\

While several computational approaches allow for an exploration of mutant proteins, they are frequently limited to a single mutation at a time.  One technique to understand a mutant protein, would be to view a 3D structure in relation to the wild type. This however is limiting, in that human understanding from a structral visualization is limited, and finding differences between a multitude of mutant proteins is impossible. Other techniques involve exploring protein folding comparatively between the wild type and a mutant protein, but this data can be difficult to interpret when the differences are small \cite{b4}. This also does not allow for some understanding and comparative analysis in regards to an exhaustive mutation screen.\\ Our technique combines the novel approach of rigidity analysis, exhaustive in-silico mutation data, and a comparative metric combined into a powerful visualization dashboard to provide a unique and powerful insight into the changes between various mutations.

Applying rigidity analysis to a protein allows us to gain insight into the structure of the protein. Since structure is crucial to the function of a protein, this insight can be of remarkable use. Rigidity analysis models the biomolecule as a mechanical structure, by identifying stabilizing interactions, and identifying rigid units. Then from the mechanical model, an associated graph is developed, and an efficient algorithm allows us to infer the rigid and flexible regions of this graph. This approach does not require costly energy calculations as is required by some other techniques, or require large datasets as is necessary for machine learning methods. Rigidity analysis allows for insight into protein stability, flexibility, and chemical clustering that may guide a deeper understanding for further exploration in a wet lab environment.\\

For this work, we have gathered computed rigidity analysis data for six proteins of various sizes and provide an interactive dashboard that allows for the exploration of mutations in regards to the wild type and comparatively for the exhaustive mutation screening.



\section{Methods}
\subsection{Data}\label{AA}
The data collected is the result of output from KINARI \cite{b1}
\subsection{Visualization Tools}\label{AA}

\section{Visualizations}
\subsection{Heatmap}\label{AA}
\subsection{Bar Chart}\label{AA}


\section{Findings}


\section{Discussion}

\section{Conclusions}

%\begin{figure}[htbp]
%\centerline{\includegraphics{fig1.png}}
%\caption{Example of a figure caption.}
%\label{fig}
%\end{figure}

%\cite{b1}

\begin{thebibliography}{00}
\bibitem{b1} N. Fox, F. Jagodzinski, and I. Streinu. "Kinari-lib: a C++ library for pebble game rigidity analysis of mechanical models." In Minisymposium on Publicly Available Geometric/Topological Software, Chapel Hill, NC, USA, June 2012.
\bibitem{b2} Q. Wang, R. Chaerkady, J. Wu, H. Hwang, N. Papadopoulos, L. Kopelovich, et al. (2011). "Mutant proteins as cancer-specific biomarkers". Proceedings of the National Academy of Sciences. 2011
\bibitem{b3} T. Maximova, R. Moatt, B. Ma, R. Nussinov, and A. Shehu. "Principles and overview of sampling methods for modeling macromolecular structure and dynamics." PLoS computational biology. 2016.
\bibitem{b4} A. Oliveira Jr., F. Fatore, F. Paulovich, O. Oliveira Jr., and V. Leite  "Visualization of Protein Folding Funnels in Lattice Models." PLoS computational biology. 2014.


\end{thebibliography}



\end{document}
